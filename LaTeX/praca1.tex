\documentclass[12pt, twoside, openany]{report}
\usepackage[dvips]{graphicx,color,rotating}
\usepackage[utf8]{inputenc}
\usepackage{t1enc}
\usepackage{a4wide}
\usepackage{amsfonts}
\usepackage{amsmath}
\usepackage{enumerate}
\usepackage{verbatim}
\usepackage[MeX]{polski}
\usepackage[T1]{fontenc}
\usepackage{geometry}
\geometry{left=25mm,right=25mm,%
bindingoffset=10mm, top=25mm, bottom=25mm}
\usepackage{amssymb, latexsym}
\usepackage{amsthm}
\usepackage{palatino}
\usepackage{array}
\usepackage{pstricks}
\usepackage{textcomp}
\usepackage{hyperref}
%paginy
\usepackage{fancyhdr}
\pagestyle{fancy}
% zmiana liter w~.ywej paginie na ma.e
\renewcommand{\chaptermark}[1]{\markboth{#1}{}}
\renewcommand{\sectionmark}[1]{\markright{\thesection\ #1}}
\fancyhf{} % usu. bie..ce ustawienia pagin
\fancyhead[LE,RO]{\small\bfseries\thepage}
\fancyhead[LO]{\small\bfseries\rightmark}
\fancyhead[RE]{\small\bfseries\leftmark}
\renewcommand{\headrulewidth}{0.5pt}
\renewcommand{\footrulewidth}{0pt}
\addtolength{\headheight}{0.5pt} % pionowy odst.p na kresk.
\fancypagestyle{plain}{%
\fancyhead{} % usu. p. górne na stronach pozbawionych
% numeracji (plain)
\renewcommand{\headrulewidth}{0pt} % pozioma kreska
}
%paginy koniec

\theoremstyle{plain}
\newtheorem{tw}{Twierdzenie}[section]
\newtheorem{uw}{Uwaga}[section]
\newtheorem{defi}{Definicja}[section]
\newtheorem{alg}{Algorytm}[chapter]
\newtheorem{przyp}{Przypadek}[section]
\newtheorem{prz}{Przykład}[section]
\newtheorem{lem}[tw]{Lemat}
\newtheorem{stw}[tw]{Stwierdzenie}
\newtheorem{wn}[tw]{Wniosek}
\newtheorem{cw}{Ćwiczenie}[section]
\linespread{1.5}
\newcommand{\R}{\mathbb{R}}
\newcommand{\N}{\mathbb{N}}
\newcommand{\al}{\alpha}
\newcommand*{\norm}[1]{\left\Vert{#1}\right\Vert}
\newcommand*{\abs}[1]{\left\vert{#1}\right\vert}
\newcommand*{\om}{\omega}

\author{Natalia Potocka}
\title{Automatyczna kategoryzacja tematyczna tekstów przy użyciu metryk w przestrzeni ciągów znaków}

\begin{document}


\begin{titlepage}
\pagestyle{headings}

\noindent
\begin{Large}
\begin{table}[t]
\centering
\begin{tabular}[t]{lcr}
 \includegraphics[width=70pt,height=70pt]{PW.png} & POLITECHNIKA WARSZAWSKA & \includegraphics[width=70pt,height=70pt]{MiNI.png}\\
& WYDZIAŁ MATEMATYKI & \\
& I NAUK INFORMACYJNYCH &
\end{tabular}
\end{table}

% \vfill
\begin{center}PRACA DYPLOMOWA MAGISTERSKA \end{center}
\begin{center}MATEMATYKA\end{center}\end{Large}
% \vfill
\begin{center}
\Huge
\textbf{Automatyczna kategoryzacja tematyczna tekstów przy użyciu metryk w przestrzeni ciągów znaków}
\end{center}
% \vfill\vfill
\vfill
\begin{center}
\Large
Autor: \\
\LARGE
Natalia Potocka
\end{center}
\vfill
\begin{center}
\Large
Promotor: \\
\Large
dr Marek Gągolewski
\end{center}
\vfill
\begin{center}
\large
Warszawa, Wrzesień 2015
\end{center}
\newpage
\hfill
\begin{table}[b]
\centering
\begin{tabular}[t]{ccc}
............................................. & \hspace*{100pt} & .............................................\\
podpis promotora & \hspace*{100pt} & podpis autora
\end{tabular}
\end{table}


% \maketitle
\end{titlepage}
\thispagestyle{empty}
\newpage
\pagestyle{headings}
\setcounter{page}{1}
\hyphenation{Syl-ves-tra}
\hyphenation{Syl-ves-ter-a}
\cleardoublepage

\tableofcontents

\begin{abstract}

\end{abstract}


%-----------Początek części zasadniczej-----------


\chapter*{Wstęp}
\addcontentsline{toc}{chapter}{Wstęp}
COŚ tu będzie.




%-----------Koniec części zasadniczej-----------

\begin{thebibliography}{11}
%\bibitem[1]{M} Arnold W. Miller, \emph{Special Subsets of the Real Line}, The University of Texas, Austin, U.S.A. 1984.
%\bibitem[2]{H2} J.C. Oxtoby, \emph{Measure and Category}, Springer-Verlag, New York Heidelberg Berlin, 1971.
%\bibitem[3]{WW} Winfried Just and Martin Weese, \emph{Discovering Modern Set Theory. I: The Basics.} Graduate Studies in Mathematics vol. 8, American Mathematical Society, Providence, RI, 1996.
%\bibitem[4]{WW2} Winfried Just and Martin Weese, \emph{Discovering Modern Set Theory. II: Set-Theoretic Tools for Every Mathematician.}, Graduate Studies in Mathematics vol. 18, American Mathematical Society, Providence, RI, 1997.
\end{thebibliography}
\clearpage
\pagestyle{empty}
\noindent Warszawa, dnia ...............
\vspace{5cm}
\begin{center}
\LARGE{Oświadczenie}
\end{center}
Oświadczam, że pracę licencjacką pod tytułem: ,,Automatyczna kategoryzacja tematyczna tekstów przy użyciu metryk w przestrzeni ciągów znaków'', której promotorem jest dr Marek Gągolewski, wykonałem/am samodzielnie, co poświadczam własnoręcznym podpisem.
\vspace{2cm}
\begin{flushright}
...........................................
\end{flushright}

\end{document}
