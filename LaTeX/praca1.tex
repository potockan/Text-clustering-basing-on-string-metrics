\documentclass[12pt, twoside, openany]{report}
\usepackage[dvips]{graphicx,color,rotating}
\usepackage[utf8]{inputenc}
\usepackage{t1enc}
\usepackage{a4wide}
\usepackage{amsfonts}
\usepackage{amsmath}
\usepackage{enumerate}
\usepackage{verbatim}
\usepackage[MeX]{polski}
\usepackage[T1]{fontenc}
\usepackage{geometry}
\geometry{left=25mm,right=25mm,%
bindingoffset=10mm, top=25mm, bottom=25mm}
\usepackage{amssymb, latexsym}
\usepackage{amsthm}
\usepackage{palatino}
\usepackage{array}
\usepackage{pstricks}
\usepackage{textcomp}
\usepackage{hyperref}
%paginy
\usepackage{fancyhdr}
\pagestyle{fancy}
% zmiana liter w~.ywej paginie na ma.e
\renewcommand{\chaptermark}[1]{\markboth{#1}{}}
\renewcommand{\sectionmark}[1]{\markright{\thesection\ #1}}
\fancyhf{} % usu. bie..ce ustawienia pagin
\fancyhead[LE,RO]{\small\bfseries\thepage}
\fancyhead[LO]{\small\bfseries\rightmark}
\fancyhead[RE]{\small\bfseries\leftmark}
\renewcommand{\headrulewidth}{0.5pt}
\renewcommand{\footrulewidth}{0pt}
\addtolength{\headheight}{0.5pt} % pionowy odst.p na kresk.
\fancypagestyle{plain}{%
\fancyhead{} % usu. p. górne na stronach pozbawionych
% numeracji (plain)
\renewcommand{\headrulewidth}{0pt} % pozioma kreska
}
%paginy koniec

\theoremstyle{plain}
\newtheorem{tw}{Twierdzenie}[section]
\newtheorem{uw}{Uwaga}[section]
\newtheorem{defi}{Definicja}[section]
\newtheorem{alg}{Algorytm}[chapter]
\newtheorem{przyp}{Przypadek}[section]
\newtheorem{prz}{Przykład}[section]
\newtheorem{lem}[tw]{Lemat}
\newtheorem{stw}[tw]{Stwierdzenie}
\newtheorem{wn}[tw]{Wniosek}
\newtheorem{cw}{Ćwiczenie}[section]
\linespread{1.5}
\newcommand{\R}{\mathbb{R}}
\newcommand{\N}{\mathbb{N}}
\newcommand{\al}{\alpha}
\newcommand*{\norm}[1]{\left\Vert{#1}\right\Vert}
\newcommand*{\abs}[1]{\left\vert{#1}\right\vert}
\newcommand*{\om}{\omega}

\author{Natalia Potocka}
\title{Automatyczna kategoryzacja tematyczna tekstów przy użyciu metryk w przestrzeni ciągów znaków}

\begin{document}
%
%
%\begin{titlepage}
%\pagestyle{headings}
%
%\noindent
%\begin{Large}
%\begin{table}[t]
%\centering
%\begin{tabular}[t]{lcr}
% \includegraphics[width=70pt,height=70pt]{PW.png} & POLITECHNIKA WARSZAWSKA & \includegraphics[width=70pt,height=70pt]{MiNI.png}\\
%& WYDZIAŁ MATEMATYKI & \\
%& I NAUK INFORMACYJNYCH &
%\end{tabular}
%\end{table}
%
%% \vfill
%\begin{center}PRACA DYPLOMOWA MAGISTERSKA \end{center}
%\begin{center}MATEMATYKA\end{center}\end{Large}
%% \vfill
%\begin{center}
%\Huge
%\textbf{Automatyczna kategoryzacja tematyczna tekstów przy użyciu metryk w przestrzeni ciągów znaków}
%\end{center}
%% \vfill\vfill
%\vfill
%\begin{center}
%\Large
%Autor: \\
%\LARGE
%Natalia Potocka
%\end{center}
%\vfill
%\begin{center}
%\Large
%Promotor: \\
%\Large
%dr Marek Gągolewski
%\end{center}
%\vfill
%\begin{center}
%\large
%Warszawa, Wrzesień 2015
%\end{center}
%\newpage
%\hfill
%\begin{table}[b]
%\centering
%\begin{tabular}[t]{ccc}
%............................................. & \hspace*{100pt} & .............................................\\
%podpis promotora & \hspace*{100pt} & podpis autora
%\end{tabular}
%\end{table}
%
%
%% \maketitle
%\end{titlepage}
%\thispagestyle{empty}
%\newpage
%\pagestyle{headings}
%\setcounter{page}{1}
%\hyphenation{Syl-ves-tra}
%\hyphenation{Syl-ves-ter-a}
%\cleardoublepage
%
%\tableofcontents
%
%\begin{abstract}
%
%\end{abstract}


%-----------Początek części zasadniczej-----------

%
%\chapter*{Wstęp}
%\addcontentsline{toc}{chapter}{Wstęp}
%COŚ tu będzie.

\chapter{Metryki na przestrzeni ciągów
znaków}\label{metryki-na-przestrzeni-ciagow-znakow}

\section{Podstawowe definicje}

\begin{defi}
\emph{Napisem} nazywamy skończone złączenie symboli (znaków) ze~skończonego \emph{alfabetu}, oznaczonego przez $\Sigma$. Produkt kartezjański rzędu $q$, $\Sigma\times\ldots\times\Sigma$ oznaczamy przez $\Sigma^q$, natomiast zbiór wszystkich skończonych napisów, które można utworzyć ze~znaków z $\Sigma$ oznaczamy przez $\Sigma^*$. \emph{Pusty napis}, oznaczany $\varepsilon$, również należy do~$\Sigma^*$. Napisy zwyczajowo będziemy oznaczać przez $s$,~$t$~oraz $u$,~a~ich \emph{długość}, czyli liczbę znaków w napisie, przez $|s|$.
\end{defi}


\begin{prz}
Niech $\Sigma$ będzie alfabetem złożonym z $26$ małych liter alfabetu łacińskiego oraz niech $s = 'ala'$. Wówczas mamy $|s| = 3$, $s \in \Sigma^3$ oraz $s \in \Sigma$. Pojedyncze znaki oznaczamy przez indeks dolny, stąd mamy $s_1 = 'a'$, $s_2 = 'l'$, $s_3 = 'a'$. %Podnapis oznaczamy przez $m:n$ w indeksie dolnym, np. $s_{1:2} = 'al'$. Jeśli $n < m$, to $s_{m:n} = \varepsilon$, czyli napis pusty.
\cite{Loo2014:stringdist}
\end{prz}

\begin{defi}
Funkcję $d$ nazywamy \emph{metryką} na~$\Sigma^*$, jeśli ma~poniższe własności:
\begin{itemize}
\item $d(s,t) \geq 0$
\item $d(s,t) = 0$ wtedy i tylko wtedy, gdy $s = t$
\item $d(s,t) = d(t,s)$
\item $d(s,u) \leq d(s,t) + d(t,u)$,
\end{itemize}
gdzie $s$,~$t$,~$u$~są~napisami.
\end{defi}


Metryki na napisach można podzielić na~trzy grupy:
\begin{itemize}
\item oparte na~operacjach edycyjnych (\emph{edit operations}),
\item oparte na~$q$-gramach,
\item miary heurystyczne.
\end{itemize}


\section{Odległości na napisach oparte na operacjach edycyjnych}

Metryki oparte na~operacjach edycyjnych zliczają liczbę opercji potrzebnych do~przetworzenia jednego napisu w~drugi. Najczęściej wymieniamymi operacjami~są:
\begin{itemize}
\item zamiana znaku, np. $'ala' \rightarrow 'ela'$
\item usunięcie znaku, np. $'ala' \rightarrow 'aa'$
\item wstawienie znaku, np. $'ala' \rightarrow 'alka'$
\item transpozycja dwóch przylegających znaków, np. $'ala' \rightarrow 'laa'$
\end{itemize}

Przykładowe odległości: Hamminga, najdłuższego wspólnego podnapisu (\emph{longest common substring}), Levenshteina, optymalnego dopasowania napisów (\emph{optimal string alignment}), Damareu-Levenshteina. Nie wszystkie z ww. odległości są metrykami.



Metryka \textbf{najdłuższego wspólnego podnapisu}, ozn. $d_{lcs}$, zlicza liczbę usunięć i~wstawień, potrzebnych do~przetworzenia jednego napisu w~drugi. Np. $d_{lsc}('leia', 'leela') = 3$, bo~$leela  \xrightarrow{us.\ e} lela  \xrightarrow{us.\ l} lea  \xrightarrow{wst.\ i} leia$.\\

Uogólniona \textbf{odległość Levenshteina}, ozn. $d_{lv}$ zlicza ważoną sumę usunięć, wstawień oraz zamian znaków, potrzebnych do~przetworzenia jednego napisu w~drugi. \\


Gdy za~wagi przyjmuje się $1$ mamy do~czynienia ze~zwykłą odległością Levenshteina, np. \\
$d_{lv}('leia', 'leela') = 2$, bo $leela  \xrightarrow{us.\ e} lela  \xrightarrow{zm.\ l\ na\ i} leia$. \\

Gdy za~wagi przyjmiemy np. $(0.1, 1, 1)$, \\
$d_{lv}('leia', 'leela') = 1.1$, bo~$leela  \xrightarrow[0.1]{us.\ e} lela  \xrightarrow[1]{zm.\ l\ na\ i} leia$ 


%-----------Koniec części zasadniczej-----------

%\begin{thebibliography}{11}
%\nocite{Hornik2012:sphkmeans}
%\nocite{Wild2002:sphkmeans}
%\nocite{Loo2014:stringdist}
\bibliographystyle{plain}
\bibliography{bibliography}
%\bibitem[1]{M} Arnold W. Miller, \emph{Special Subsets of the Real Line}, The University of Texas, Austin, U.S.A. 1984.
%\bibitem[2]{H2} J.C. Oxtoby, \emph{Measure and Category}, Springer-Verlag, New York Heidelberg Berlin, 1971.
%\bibitem[3]{WW} Winfried Just and Martin Weese, \emph{Discovering Modern Set Theory. I: The Basics.} Graduate Studies in Mathematics vol. 8, American Mathematical Society, Providence, RI, 1996.
%\bibitem[4]{WW2} Winfried Just and Martin Weese, \emph{Discovering Modern Set Theory. II: Set-Theoretic Tools for Every Mathematician.}, Graduate Studies in Mathematics vol. 18, American Mathematical Society, Providence, RI, 1997.
%\end{thebibliography}
\clearpage
\pagestyle{empty}
\noindent Warszawa, dnia ...............
\vspace{5cm}
\begin{center}
\LARGE{Oświadczenie}
\end{center}
Oświadczam, że pracę licencjacką pod tytułem: ,,Automatyczna kategoryzacja tematyczna tekstów przy użyciu metryk w przestrzeni ciągów znaków'', której promotorem jest dr Marek Gągolewski, wykonałem/am samodzielnie, co poświadczam własnoręcznym podpisem.
\vspace{2cm}
\begin{flushright}
...........................................
\end{flushright}

\end{document}
