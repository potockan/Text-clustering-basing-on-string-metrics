\documentclass[]{article}
\usepackage[T1]{fontenc}
\usepackage{lmodern}
\usepackage{amssymb,amsmath}
\usepackage{ifxetex,ifluatex}
\usepackage{fixltx2e} % provides \textsubscript
% Set line spacing
% use upquote if available, for straight quotes in verbatim environments
\IfFileExists{upquote.sty}{\usepackage{upquote}}{}
\ifnum 0\ifxetex 1\fi\ifluatex 1\fi=0 % if pdftex
  \usepackage[utf8]{inputenc}
\else % if luatex or xelatex
  \ifxetex
    \usepackage{mathspec}
    \usepackage{xltxtra,xunicode}
  \else
    \usepackage{fontspec}
  \fi
  \defaultfontfeatures{Mapping=tex-text,Scale=MatchLowercase}
  \newcommand{\euro}{€}
\fi
% use microtype if available
\IfFileExists{microtype.sty}{\usepackage{microtype}}{}
\usepackage[margin=1in]{geometry}
\ifxetex
  \usepackage[setpagesize=false, % page size defined by xetex
              unicode=false, % unicode breaks when used with xetex
              xetex]{hyperref}
\else
  \usepackage[unicode=true]{hyperref}
\fi
\hypersetup{breaklinks=true,
            bookmarks=true,
            pdfauthor={Natalia Potocka},
            pdftitle={Metryki na przestrzeni ciągów znaków},
            colorlinks=true,
            citecolor=blue,
            urlcolor=blue,
            linkcolor=magenta,
            pdfborder={0 0 0}}
\urlstyle{same}  % don't use monospace font for urls
\setlength{\parindent}{0pt}
\setlength{\parskip}{6pt plus 2pt minus 1pt}
\setlength{\emergencystretch}{3em}  % prevent overfull lines
\setcounter{secnumdepth}{0}

%%% Change title format to be more compact
\usepackage{titling}
\setlength{\droptitle}{-2em}
  \title{Metryki na przestrzeni ciągów znaków}
  \pretitle{\vspace{\droptitle}\centering\huge}
  \posttitle{\par}
  \author{Natalia Potocka}
  \preauthor{\centering\large\emph}
  \postauthor{\par}
  \predate{\centering\large\emph}
  \postdate{\par}
  \date{Luty, 2015}




\begin{document}

\maketitle


\section{Metryki na przestrzeni ciągów
znaków}\label{metryki-na-przestrzeni-ciagow-znakow}

Def. \emph{Alfabetem} nazywamy niepusty zbiór \emph{liter}. Oznaczmy go
przez $\Sigma$. Przez \emph{słowo} nad alfabetem $\Sigma$ rozumiemy
skończony ciąg elementów z $\Sigma$. Zbiór wszystkich słów z $\Sigma$
oznaczamy przez $\Sigma^*$. \emph{Językiem} nad $\Sigma$ nazywamy każdy
podzbiór $\Sigma^*$.

Def. Pusty ciąg znaków, oznaczony przez $\varepsilon$, jest nazywamy
\emph{pustym napisem}. Dla słowa $w$, $|w|$ oznacza długość $w$. Dla
każdego $a \in \Sigma$, $|w|_a$ oznacza liczba wystąpień $a$ w $w$.Dla
każdego $i \in {1,2, \ldots, |w|}$, $w[i]$ oznacza $i$-tą literę z $w$.
Mając dane dwa słowa, $x$ i $y$, poprzez $xy$ rozumiemy \emph{złączenie}
$x$ i $y$. Dla każdego $n \in \mathbb{N}$, definiujemy $x^n$ jako $n$-tą
potęgę $x$, to znaczy złączenie $n$ kopii słowa $x$ (zauważmy, że
$x^0=\varepsilon$). Dla każdego $L \subseteq \Sigma^*$ i dla każdego
$w \in \Sigma^*$ , oznaczamy $Lw = {xw: x \in L}$.

\end{document}
