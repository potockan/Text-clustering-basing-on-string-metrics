\documentclass[12pt, twoside, openany]{report}\usepackage[]{graphicx}\usepackage[]{color}
%% maxwidth is the original width if it is less than linewidth
%% otherwise use linewidth (to make sure the graphics do not exceed the margin)
\makeatletter
\def\maxwidth{ %
  \ifdim\Gin@nat@width>\linewidth
    \linewidth
  \else
    \Gin@nat@width
  \fi
}
\makeatother

\definecolor{fgcolor}{rgb}{0.345, 0.345, 0.345}
\newcommand{\hlnum}[1]{\textcolor[rgb]{0.686,0.059,0.569}{#1}}%
\newcommand{\hlstr}[1]{\textcolor[rgb]{0.192,0.494,0.8}{#1}}%
\newcommand{\hlcom}[1]{\textcolor[rgb]{0.678,0.584,0.686}{\textit{#1}}}%
\newcommand{\hlopt}[1]{\textcolor[rgb]{0,0,0}{#1}}%
\newcommand{\hlstd}[1]{\textcolor[rgb]{0.345,0.345,0.345}{#1}}%
\newcommand{\hlkwa}[1]{\textcolor[rgb]{0.161,0.373,0.58}{\textbf{#1}}}%
\newcommand{\hlkwb}[1]{\textcolor[rgb]{0.69,0.353,0.396}{#1}}%
\newcommand{\hlkwc}[1]{\textcolor[rgb]{0.333,0.667,0.333}{#1}}%
\newcommand{\hlkwd}[1]{\textcolor[rgb]{0.737,0.353,0.396}{\textbf{#1}}}%

\usepackage{framed}
\makeatletter
\newenvironment{kframe}{%
 \def\at@end@of@kframe{}%
 \ifinner\ifhmode%
  \def\at@end@of@kframe{\end{minipage}}%
  \begin{minipage}{\columnwidth}%
 \fi\fi%
 \def\FrameCommand##1{\hskip\@totalleftmargin \hskip-\fboxsep
 \colorbox{shadecolor}{##1}\hskip-\fboxsep
     % There is no \\@totalrightmargin, so:
     \hskip-\linewidth \hskip-\@totalleftmargin \hskip\columnwidth}%
 \MakeFramed {\advance\hsize-\width
   \@totalleftmargin\z@ \linewidth\hsize
   \@setminipage}}%
 {\par\unskip\endMakeFramed%
 \at@end@of@kframe}
\makeatother

\definecolor{shadecolor}{rgb}{.97, .97, .97}
\definecolor{messagecolor}{rgb}{0, 0, 0}
\definecolor{warningcolor}{rgb}{1, 0, 1}
\definecolor{errorcolor}{rgb}{1, 0, 0}
\newenvironment{knitrout}{}{} % an empty environment to be redefined in TeX

\usepackage{alltt}
\usepackage[dvips]{rotating}
\usepackage[utf8]{inputenc}
\usepackage{t1enc}
\usepackage{a4wide}
\usepackage{amsfonts}
\usepackage{amsmath}
\usepackage{enumerate}
\usepackage{verbatim}
\usepackage[MeX]{polski}
\usepackage[T1]{fontenc}
\usepackage{geometry}
\geometry{left=25mm,right=25mm,%
bindingoffset=10mm, top=25mm, bottom=25mm}
\usepackage{amssymb, latexsym}
\usepackage{amsthm}
\usepackage{palatino}
\usepackage{array}
\usepackage{pstricks}
\usepackage{textcomp}
\usepackage{hyperref}
%paginy
\usepackage{fancyhdr}
\pagestyle{fancy}
% zmiana liter w~.ywej paginie na ma.e
\renewcommand{\chaptermark}[1]{\markboth{#1}{}}
\renewcommand{\sectionmark}[1]{\markright{\thesection\ #1}}
\fancyhf{} % usu. bie..ce ustawienia pagin
\fancyhead[LE,RO]{\small\bfseries\thepage}
\fancyhead[LO]{\small\bfseries\rightmark}
\fancyhead[RE]{\small\bfseries\leftmark}
\renewcommand{\headrulewidth}{0.5pt}
\renewcommand{\footrulewidth}{0pt}
\addtolength{\headheight}{0.5pt} % pionowy odst.p na kresk.
\fancypagestyle{plain}{%
\fancyhead{} % usu. p. górne na stronach pozbawionych
% numeracji (plain)
\renewcommand{\headrulewidth}{0pt} % pozioma kreska
}
%paginy koniec

\theoremstyle{plain}
\newtheorem{tw}{Twierdzenie}[section]
\newtheorem{uw}{Uwaga}[section]
\newtheorem{defi}{Definicja}[section]
\newtheorem{alg}{Algorytm}[chapter]
\newtheorem{przyp}{Przypadek}[section]
\newtheorem{prz}{Przykład}[section]
\newtheorem{lem}[tw]{Lemat}
\newtheorem{stw}[tw]{Stwierdzenie}
\newtheorem{wn}[tw]{Wniosek}
\newtheorem{cw}{Ćwiczenie}[section]
\linespread{1.5}


\author{Natalia Potocka}
\title{Słowa - statystyki}
\IfFileExists{upquote.sty}{\usepackage{upquote}}{}
\begin{document}

\maketitle

\hyphenation{Syl-ves-tra}
\hyphenation{Syl-ves-ter-a}

\newpage







W polskiej Wikipedii znaleziono 3802968 unikalnych słów. 56.74\% z nich występuje dokładnie w jednym tekście, natomiast 51.7\% występuje tylko raz. Dobrze obrazuje to poniższy histogram.

\begin{knitrout}
\definecolor{shadecolor}{rgb}{0.969, 0.969, 0.969}\color{fgcolor}
\includegraphics[width=\maxwidth]{figure/hist-1} 

\end{knitrout}

Słowa występujące tylko w jednym tekście to zazwyczaj liczby oraz słowa w obcych językach, przykładowo:

\begin{knitrout}
\definecolor{shadecolor}{rgb}{0.969, 0.969, 0.969}\color{fgcolor}\begin{kframe}
\begin{verbatim}
 [1] "300346"        "bua2"          "przechowujesz" "168524"       
 [5] "bedoin"        "78341"         "2020551446"    "niewspomniani"
 [9] "terdsakiem"    "anamorf"      
\end{verbatim}
\end{kframe}
\end{knitrout}

choć czasem są to słowa będące odmianą słów bardziej częściej występujących, np. słowo \emph{przechowujesz} jest odmianą czasownika \emph{przechowywać}. Należy wziąć pod uwagę takie słowa, gdyż mogą one polepszyć jakość dopasowania tekstów pod względem tematycznym.

% Po usunięciu słów występujących w jednym, dwóch lub trzech tekstach otrzymujemy następujące statystyki:
% 
% <<stat_3, dependson='ws'>>=
% cnt <- 3
% stat_cnt <- which(word_stat$word_cnt>cnt)
% print(summary(word_stat$word_cnt[stat_cnt]))
% n_stat_3 <- length(stat_cnt)/n_words*100
% @
% 
% 
% Słowa, które występują przynajmniej w czterech stanowią zaledwie n_stat_3\% wszystkich słów.

Słowa występujące w największej liczbie tekstów to:\\

% latex table generated in R 3.1.2 by xtable 1.7-4 package
% Tue Dec 30 15:58:22 2014
\begin{tabular}{rlrr}
  \hline
 & Słowo & Liczba artykułów & Liczba wystąpień słowa \\ 
  \hline
1 & w & 1001499 & 10246420 \\ 
  2 & i & 723899 & 4240658 \\ 
  3 & na & 687441 & 3183731 \\ 
  4 & z & 647318 & 3218341 \\ 
  5 & do & 557602 & 2273872 \\ 
  6 & się & 535273 & 2066420 \\ 
  7 & roku & 418980 & 1284882 \\ 
  8 & a & 377247 & 908846 \\ 
  9 & od & 376645 & 926187 \\ 
  10 & jest & 342055 & 977374 \\ 
   \hline
\end{tabular}


%%%%%%%%%%%%%%%%% TO DO %%%%%%%%%%%%%%%%%%%%%%
%%%% wgrac (slowo, licznosc slowa ogolem) %%%%

W większości przypadków są to słowa nie istotne w kontekście analizy tematycznej tekstu. Podobnych słów jest więcej. Ich pełna lista zostaje przedstawiona poniżej:

\begin{knitrout}
\definecolor{shadecolor}{rgb}{0.969, 0.969, 0.969}\color{fgcolor}\begin{kframe}
\begin{verbatim}
  [1] "ach"         "aj"          "albo"        "bardzo"      "bez"        
  [6] "bo"          "być"         "ci"          "cię"         "ciebie"     
 [11] "co"          "czy"         "daleko"      "dla"         "dlaczego"   
 [16] "dlatego"     "do"          "dobrze"      "dokąd"       "dość"       
 [21] "dużo"        "dwa"         "dwaj"        "dwie"        "dwoje"      
 [26] "dziś"        "dzisiaj"     "gdyby"       "gdzie"       "go"         
 [31] "ich"         "ile"         "im"          "inny"        "ja"         
 [36] "ją"          "jak"         "jakby"       "jaki"        "je"         
 [41] "jeden"       "jedna"       "jedno"       "jego"        "jej"        
 [46] "jemu"        "jeśli"       "jest"        "jestem"      "jeżeli"     
 [51] "już"         "każdy"       "kiedy"       "kierunku"    "kto"        
 [56] "ku"          "lub"         "ma"          "mają"        "mam"        
 [61] "mi"          "mną"         "mnie"        "moi"         "mój"        
 [66] "moja"        "moje"        "może"        "mu"          "my"         
 [71] "na"          "nam"         "nami"        "nas"         "nasi"       
 [76] "nasz"        "nasza"       "nasze"       "natychmiast" "nią"        
 [81] "nic"         "nich"        "nie"         "niego"       "niej"       
 [86] "niemu"       "nigdy"       "nim"         "nimi"        "niż"        
 [91] "obok"        "od"          "około"       "on"          "ona"        
 [96] "one"         "oni"         "ono"         "owszem"      "po"         
[101] "pod"         "ponieważ"    "przed"       "przedtem"    "są"         
[106] "sam"         "sama"        "się"         "skąd"        "tak"        
[111] "taki"        "tam"         "ten"         "to"          "tobą"       
[116] "tobie"       "tu"          "tutaj"       "twoi"        "twój"       
[121] "twoja"       "twoje"       "ty"          "wam"         "wami"       
[126] "was"         "wasi"        "wasz"        "wasza"       "wasze"      
[131] "we"          "więc"        "wszystko"    "wtedy"       "wy"         
[136] "żaden"       "zawsze"      "że"         
\end{verbatim}
\end{kframe}
\end{knitrout}
% 
% <<rm_stp>>=
% words <- data.frame(word=setdiff(word_stat$word[stat_cnt], stopwords))
% word_no_stp <- merge(word_stat, words)
% n_no_stp <- nrow(words)
% @
% 
% 
% Po usunięciu wyżej wymienionych słów pozostaje n_no_stp unikalnych słów do analizy.
% 
% <<hist_stp>>=
% ggplot(word_no_stp, aes(x=word)) +
%   geom_histogram(fill="white", colour="black")
% print(summary(word_no_stp$word_cnt)
% @

\end{document}
