\documentclass{praca1}
\usepackage[utf8]{inputenc}
%%------------------------------------------------------------------------------%

%------------------------------------------------------------------------------%

%\usepackage[dvips]{graphicx,color,rotating}
%\usepackage[utf8]{inputenc}
%\usepackage{t1enc}
%\usepackage{a4wide}
%\usepackage{amsfonts}
\usepackage{amsmath}
\usepackage{examplep}
\usepackage{xcolor}
%%\usepackage{enumitem}
%\usepackage{enumerate}
%\usepackage{verbatim}
%\usepackage[MeX]{polski}
%\usepackage[T1]{fontenc}
%\usepackage{geometry}
%\geometry{left=25mm,right=25mm,%
%bindingoffset=10mm, top=25mm, bottom=25mm}
%\usepackage{amssymb, latexsym}
%\usepackage{amsthm}
%\usepackage{palatino}
%\usepackage{array}
%\usepackage{pstricks}
%\usepackage{textcomp}
%\usepackage{hyperref}
%%paginy
%\usepackage{fancyhdr}





\author{Natalia Potocka}
\title{Automatyczna kategoryzacja tematyczna tekstów przy użyciu metryk w przestrzeni ciągów znaków}
\supervisor{dr Marek Gągolewski}
\type{magisters}
\discipline{matematyka}
\monthyear{grudzień 2015}
\date{\today}
\album{237476}

\begin{document}
%\maketitle
%\tableofcontents

%-----------Początek części zasadniczej-----------

%
%\chapter*{Wstęp}
%\addcontentsline{toc}{chapter}{Wstęp}
%COŚ tu będzie.

\chapter{Odległości na przestrzeni ciągów znaków}\label{metryki-na-przestrzeni-ciagow-znakow}
%\color{red}

\section{Podstawowe definicje}

CZY TO JAKO CIĄGŁY TEKST CZY WSZYSTKO PISAĆ W DEFINICJI???

%\begin{definition}
Niech  $\Sigma = \{\Sigma_i\}$ będzie skończonym uporządkowanym alfabetem o~wielkości $|\Sigma|$.~\emph{Napisem} nazywamy skończony ciąg znaków z~$\Sigma$.~Zbiór wszystkich napisów o~długości $n$~nad $\Sigma$~jest oznaczony przez $\Sigma^n$,~podczas gdy przez $\Sigma^* = \bigcup_{n=1}^{\infty}\Sigma^n$ rozumiemy zbiór wszystkich napisów utworzonych ze~znaków z~$\Sigma$~\cite{Boytsov2011:indexingmethods}.

O~ile nie podano inaczej, używamy zmiennych $s,\ t,\ u,\ v,\ w,\ x,\ y$ jako oznaczenie napisów oraz $a,\ b,\ c$ do oznaczenia napisów jednoznakowych albo po~prostu \emph{znaków}. Pusty napis jest oznaczany przez $\varepsilon$. Przez $|s|$, dla każdego napisu $s \in \Sigma^*$, rozumiemy jego długość, czyli liczbę znaków w~napisie. Ciąg zmiennych oznaczających napisy i/lub znaki oznaczają ich złączenie~\cite{Boytsov2011:indexingmethods}. Dla rozróżnienia napisów od~zmiennych reprezentujących napis, te~pierwsze oznaczamy pismem maszynowym, np.~\verb|napis|.

Poprzez $s_i$~rozumiemy $i$-ty znak z~napisu $s$,~dla każdego $i \in \{1,\ldots,|s|\}$. Podciąg kolejnych przylegających do~siebie znaków z~napisu nazywamy \emph{podnapisem}. Podnapisem napisu $s$,~który zaczyna się od~$i$-tego znaku, a~kończy na~$j$-tym znaku, oznaczamy przez $s_{i:j}$, tj. $s_{i:j} = s_is_{i+1}\ldots s_j$ dla $i<j$. Zakładamy również, że~jeśli $j<i$, to $s_{i:j} = \varepsilon$~\cite{Boytsov2011:indexingmethods,Loo2014:stringdist}.

Załóżmy, że~napis $s$~jest resprezentacją złączenia trzech, być może pustych, podnapisów $w$, $x$ i $y$, tj. $s = wxy$. Wówczas podnapis $w$ nazywamy \emph{prefiksem}, natomiast podnapis $y$ -- sufiksem~\cite{Boytsov2011:indexingmethods}.

Podnapis złożony z kolejnych znaków napisu, o ustalonej długości $q$ jest nazywany \emph{$q$-gramem}. $q$-gramy o $q$ równym jeden, dwa lub trzy mają specjalne nazwy: \emph{unigram, bigram} i \emph{trigram}. Jeśli $q > |s|$, to $q$-gramy napisu $s$ są napisami pustymi~\cite{Boytsov2011:indexingmethods}.
%\end{definition}

%\begin{definition}
%\emph{Napisem} nazywamy skończone złączenie symboli (znaków) ze~skończonego \emph{alfabetu}, oznaczonego przez $\Sigma$. Produkt kartezjański rzędu $q$, $\Sigma\times\ldots\times\Sigma$ oznaczamy przez $\Sigma^q$, natomiast zbiór wszystkich skończonych napisów, które można utworzyć ze~znaków z $\Sigma$ oznaczamy przez~$\Sigma^*$. \emph{Pusty napis}, oznaczany $\varepsilon$, również należy do~$\Sigma^*$. Napisy zwyczajowo będziemy oznaczać przez $s$,~$t$~oraz $u$,~a~ich \emph{długość}, czyli liczbę znaków w~napisie, przez $|s|$. Poprzez $s_i$ rozumiemy $i$-ty znak z napisu $s$, dla każdego $i \in \{1,\ldots,|s|\}$, natomiast podnapisy od znaku $i$-tego do znaku $j$-tego oznaczamy przez $s_{i:j}$. Zakładamy również, że jeśli $j<i$, to $s_{i:j} = \varepsilon$ \cite{Loo2014:stringdist}.
%\end{definition}


\begin{example}
Niech $\Sigma$ będzie alfabetem złożonym z~$26$ małych liter alfabetu łacińskiego oraz niech $s = \verb|ela|$. Wówczas mamy $|s| = 3$, $s \in \Sigma^3$ oraz $s \in \Sigma$. Co więcej, mamy $s_1 = \verb|e|$, $s_2 = \verb|l|$, $s_3 = \verb|a|$. Podnapis $1:2$ napisu $s$ to $s_{1:2} = \verb|el|$. W napisie tym mamy do czynienia jedynie z $q$-gramami o $q$ równym jeden, dwa oraz trzy: $\verb|e|,\ \verb|l|,\ \verb|a|$; $\verb|el|,\ \verb|la|$ oraz $\verb|ela|$ odpowiednio.
\end{example}


%[TU TRZEBA BARDZIEJ FORMALNIE! BOYTSOV STR. 4-7]Odległość $d(s,t)$ pomiędzy dwoma napisami $s$~i~$t$~to~minimalny koszt ciągu operacji potrzebnego do~przetransformowania $s$~w~$t$ (i~$\infty$, gdy taki ciąg nie istnieje). Koszt ciągu operacji jest sumą kosztów pojedynczych operacji. Przez operacje rozumiemy skończoną liczbę reguł w~formie $\delta(x, y) = a$, gdzie $x$~i~$y$ to różne podnapisy, a~$a$~to~nieujemna liczba rzeczywista. Kiedy już, przy pomocy operacji, podnapis $x$~zostanie przekształcony w~napis $y$,~żadne dalsze operacje nie mogą być wykonywane na~$y$~\cite{Navarro2001:guidedtour}.

%Zauważmy w~szczególności ostatnie ograniczenie, które nie pozwala wielokrotnie przekształcać tego samego podnapisu. DO POPRAWKI!!!: Gdyby pominąć to~założenie, każdy system przekształcający napisy spełniałby definicję i~stąd odległość między dwoma napisami nie byłaby, w~ogólności, możliwa do~policzenia~\cite{Navarro2001:guidedtour}.

%Jeśli dla każdej operacji $\delta(x,y)$, istnieje odpowiednia operacja $\delta(y,x)$ o~takim samym koszcie, to~odległość jest symetryczna (tj. $d(s,t) = d(t,s)$). Zauważmy również,~że:
%\begin{itemize}
%\item $d(s,t) \geq 0$ dla wszystkich napisów $s, t$,
%\item $d(s,s) = 0$,
%\item $d(s,u) \leq d(s,t) + d(t,u)$.
%\end{itemize}
%Stąd, jeśli odległość jest symetryczna, przestrzeń napisów tworzy przestrzeń metryczną~\cite{Navarro2001:guidedtour}.
%
%\begin{definition}
%\label{def:001}
%Funkcję $d$ nazywamy \emph{metryką} na~$\Sigma^*$, jeśli ma~poniższe własności:
%\begin{enumerate}
%\item \label{def:001a} $d(s,t) \geq 0$
%\item \label{def:001b} $d(s,t) = 0$ wtedy i tylko wtedy, gdy $s = t$
%\item \label{def:001c} $d(s,t) = d(t,s)$
%\item \label{def:001d} $d(s,u) \leq d(s,t) + d(t,u)$,
%\end{enumerate}
%gdzie $s$,~$t$,~$u$~są~napisami z~$\Sigma^*$.
%\end{definition}

%Z~powyższej definicji wynika, że~dla dowolnych $s,\ t \in \Sigma^*,\ d(s, t) \geq 0$.



\section{Odległości na przestrzeni ciągów znaków}

W tym podrozdziale zajmiemy się odległościami na przestrzeni ciągów znaków. Można je podzielić na~trzy grupy:
\begin{itemize}
\item oparte na~operacjach edycyjnych (\emph{edit operations}),
\item oparte na~$q$-gramach,
\item miary heurystyczne.
\end{itemize}

BLA BLA JAKIEŚ LANIE WODY O METRYKACH
Pierwszy rodzaj odległości jest najczęściej używany w algorytmach zajmujących się optymalnym dopasowaniem, dlatego też poświęcimy mu największą część niniejszego rozdziału. Odległości oparte na $q$-gramach ... (TUTAJ COŚ O NICH). Natomiast miary heurystyczne są rzadko stosowane, będąc zazwyczaj używane w konkretnych przypadkach. Miary te miały jednak swój wkład w historię optymalizacji napisów, zatem pokrótce opiszemy je pod koniec tego rozdziału.

\subsection{Odległości oparte na operacjach edycyjnych}

HISTORIA ODLEGLOSCI EDYCYJNYCH?

\textbf{Ścieżka edycyjna i bazowe operacje edycyjne.} \emph{Odległość edycyjna} $ED(s,t)$ pomiędzy dwoma napisami $s$~i~$t$~to~minimalna liczba operacji edycyjnych potrzebna do~przetworzenia $s$~w~$t$ (i~$\infty$, gdy taki ciąg nie istnieje)~\cite{Navarro2001:guidedtour}. \emph{Ścisłą odległością edycyjną} nazywamy minimalną liczbę nie nakładających się operacji edycyjnych, które pozwalają przekształcić jeden napis w drugi, i które nie przekształcają dwa razy tego samego podnapisu~\cite{Boytsov2011:indexingmethods}.

Napis może zostać przetworzony w drugi poprzez wykonanie na nim ciągu przekształceń jego podnapisów. Ten ciąg nazywany jest \emph{ścieżką edycyjną (śladem edycji?)}, podczas gdy przekształcenia są nazywane \emph{bazowymi operacjami edycyjnymi}. Bazowe operacje edycyjne, które polegają na mapowaniu napisu $s$ w napis $t$, są oznaczane przez $s \rightarrow t$. Zbiór wszystkich bazowych operacji edycyjnych oznaczamy przez~$\mathbb{B}$~\cite{Boytsov2011:indexingmethods}.


Bazowe operacje edycyjne są zazwyczaj ograniczone do:
\begin{itemize}
\item usunięcie znaku: $\verb|l| \rightarrow \varepsilon$, tj. usunięcie litery $\verb|l|$ , np. $\verb|ela| \rightarrow \verb|ea|$,
\item wstawienie znaku: $\varepsilon \rightarrow \verb|k|$, tj. wstawienie litery $\verb|k|$, np. $\verb|ela| \rightarrow \verb|elka|$,
\item zamiana znaku: $\verb|e| \rightarrow \verb|a|$, tj. zamiana litery $\verb|e|$ na $\verb|a|$, np. $\verb|ala| \rightarrow \verb|ela|$,
\item transpozycja: $\verb|el| \rightarrow \verb|le|$, tj. przestawienie dwóch przylegających liter $\verb|e|$ i $\verb|l|$, np. $\verb|ela| \rightarrow \verb|lea|$.
\end{itemize}

%Koszt wszystkich powyższych operacji zazwyczaj wynosi $1$. Dla wszystkich odległości, które dopuszczają więcej niż jedną operację edycyjną, może być znaczące nadanie wag poszczególnym operacjom, dając na przykład transpozycji mniejszy koszt niż operacji wstawienia znaku. Odległości, dla których takie wagi zostają nadane są zazwyczaj nazywane \emph{uogólnionymi} odległościami~\cite{Boytsov2011:indexingmethods}.

\begin{property}\label{wl:001}
Zakładamy, że $\mathbb{B}$ spełnia następujące własności~\cite{Boytsov2011:indexingmethods}:
\begin{itemize}
\item jeśli $s \rightarrow t \in \mathbb{B}$, to odwrotna operacja $t \rightarrow s$ również należy do $\mathbb{B}$;
\item $a \rightarrow a \in \mathbb{B}$ (operacja identycznościowa dla jednego znaku należy do $\mathbb{B}$);
\item zbiór $\mathbb{B}$ jest zupełny: dla dwóch dowolnych napisów $s$ i $t$, istnieje ślad edycji, który przekształca $s$ w $t$.
\end{itemize}
\end{property}

Zauważmy, że zbiór $\mathbb{B}$ nie musi być skończony.

\textbf{Odległość edycyjna.} Podobieństwo dwóch napisów może być wyrażone jako długość ścieżki edycyjnej, dzięki której jeden napis zostaje przekształcony w drugi:

\begin{definition}
Mając dany zbiór bazowych operacji edycyjnych, \emph{odległość edycyjna} $ED(s,t)$ jest równa długości najkrótszej ścieżki edycyjnej, która przekształca napis $s$ w napis $t$. Najkrótsza ścieżka, która przekształca napis $s$ w napis $t$ jest nazywana \emph{optymalną} ścieżką edycyjną~\cite{Boytsov2011:indexingmethods}. 
\end{definition}

Przykładowe odległości edycyjne: Hamminga, najdłuższego wspólnego podnapisu (\emph{longest common substring}), Levenshteina, optymalnego dopasowania napisów (\emph{optimal string alignment}), Damareu-Levenshteina. Odległości te różnią się zbiorem bazowych operacji edycyjnych. Jeśli w zbiorze tym znajduje się tylko zamiana znaków, to mamy do czynienia z odległością Hamminga. Gdy zbiór bazowych operacji edycyjnych zawiera wstawienia i usunięcia znaków, to jest to odległość najdłuższego wspólnego podnapisu. Gdyby $\mathbb{B}$ powiększyć o zamianę znaków, to otrzymamy odległość Levenshteina. Dwie ostatnie odległości, tj. optymalnego dopasowania napisów oraz Damareu-Levenshteina, mają w zbiorze bazowych operacji edycyjnych usunięcie, wstawienie, zamianę oraz transpozycję znaków.

\begin{example}
JAKIŚ PRZYKŁĄD SCIEZKI I OPTYMALNEJ SCIEZKI.
\end{example}


Definicja odległości edycyjnej może być również interpretowana jako minimalny koszt, dzięki któremu przekształcamy jeden napis w drugi. Definicję można uogólnić na dwa sposoby. Po pierwsze, bazowe operacje edycyjne mogą mieć przydzielone koszty (wagi) $\delta(a \rightarrow b)$~\cite{Wagner1974:stringtostring}. Zazwyczaj koszt każdej operacji wynosi jeden, jednak można, na przykład, nadać transpozycji mniejszy koszt niż operacji wstawienia znaku. Dalej, można rozszerzyć funkcję kosztu $\delta$ na ścieżkę edycyjną $E = a_1 \rightarrow b_1, a_2 \rightarrow b_2, \ldots, a_{E|} \rightarrow b_{|E|}$ poprzez $\delta(E) = \sum\limits_{i=1}^{|E|}\delta(a_i \rightarrow b_i)$~\cite{Boytsov2011:indexingmethods}. Odtąd poprzez odległość między napisem $s$ a napisem $t$ będziemy rozumieć minimalny ze wszystkich możliwych kosztów ścieżek przekształcających $s$ w $t$. Odległości zdefiniowane w ten sposób zazwyczaj są nazwyane \emph{uogólnionymi} odległościami edycyjnymi.\\
Po drugie, zbiór operacji edycyjnych $\mathbb{B}$ może zostać rozszerzony o ważone zamiany (substytucje) (pod)napisów, zamiast operacji edycyjnych wykonywanych na pojedynczych znakach~\cite{Ukkonen1985:algorithmsforapprox}. Odległości zdefiniowane w ten sposób zazwyczaj są nazwyane \emph{rozszerzonymi} odległościami edycyjnymi. Przykładowo, $\mathbb{B}$ może zawierać operację $\verb|x| \rightarrow \verb|ks|$ o koszcie jednostkowym. Wówczas rozszerzona odległość pomiędzy napisami $\verb|xero|$ i $\verb|ksero|$ wynosi jeden, podczas gdy standardowa (zwykła, nierozszerzona) odległość wyniosłaby dwa~\cite{Boytsov2011:indexingmethods}.

\begin{definition}\label{def:002}
Mając dany zbiór bazowych operacji edycyjnych $\mathbb{B}$ oraz funkcję $\delta$,~która nadaje koszt wszystkim bazowym operacjom edycyjnym z~$\mathbb{B}$,~ogólna(?) odległość edycyjna pomiędzy napisami $s$~i~$t$~jest zdefiniowana jako minimalny koszt ścieżki edycyjnej, która przekształca $s$~w~$t$~\cite{Boytsov2011:indexingmethods}.
\end{definition}

Zazwyczaj koszt pojedynczej operacji operacji z $\mathbb{B}$ jest równy jeden. Czasem jednak nadaje się poszczególnym operacjom różne koszty, dając np. transpozycji mniejszą wagę niż wstawieniu znaku. Gdy koszt wszystkich operacji jest równy jeden, to mówimy po prostu o odległości edycyjnej, natomiast gdy różne operacje mają różne wagi, to mówimy o \emph{uogólnionej} odległości edycyjnej.

\begin{property}\label{wl:002}
Zakładamy, że funkcja kosztu $\delta(s \rightarrow t)$ ma następujące własności~\cite{Boytsov2011:indexingmethods}:
\begin{itemize}
\item $\delta(s \rightarrow t) \in \mathbb{R}$ (koszt operacji jest liczbą rzeczywistą),
\item $\delta(s \rightarrow t) = \delta(t \rightarrow s)$ (symetria),
\item $\delta(s \rightarrow t) \geq 0,\ \delta(s \rightarrow s) = 0\text{ i } \delta(s \rightarrow t) = 0 \Rightarrow s = t$ (dodatnia określoność ??),
\item $\forall \gamma > 0$ zbiór bazowych operacji $\{s \rightarrow t \in \mathbb{B} | \delta(s \rightarrow t) < \gamma \}$ jest skończony (skończoność podzbioru bazowych operacji, których koszt jest ograniczony z góry).
\end{itemize}
\end{property}

Zauważmy, że ostatnia własność jest zawsze spełniona dla skończonego zbioru $\mathbb{B}$.

\begin{theorem}
Z własności \ref{wl:001} i \ref{wl:002} wynika, że:
\begin{itemize}
\item dla każdych dwóch napisów $s$ i $t$, istnieje ścieżka o minimalnym koszcie, tj. właściwie (prawdziwie, odpowiednio?) zdefiniowana odleglość edycyjna z $s$ do $t$~\cite{Boytsov2011:indexingmethods},
\item ogólna odległość edycyjna z definicji \ref{def:002} jest metryką~\cite{Wagner1974:stringtostring}.
\end{itemize}
\end{theorem}

\begin{proof}
Żeby udowodnić, że $ED(s,t)$ jest metryką, musimy pokazać, że $ED(s,t)$ istnieje, 	jest dodatnio określona, symetryczna oraz subaddytywna (tj. spełnia nierówność trójkąta).

Z własności~\ref{wl:002} wynika, że funkcja kosztu jest nieujemna (JA TEGO NIW WIDZE) i że tylko identyczność ma koszt równy zero. Stąd, bez utraty ogólności, możemy rozważyć jedynie takie ścieżki edycyjne, które nie zawierają operacji identycznościowych. Zatem, jeśli $s=t$, to jedyna optymalna ścieżka (która nie zawiera operacji identycznościowych) jest pusta i ma zerowy koszt. Jeśli $s\neq t$, to z zupełności zbioru bazowych operacji edycyjnych wynika, że istnieje jedna lub więcej ścieżek edycyjnych, które przekształcają $s$ w $t$. Wszystkie te ścieżki składają się z operacji edycyjnych o ściśle dodatnim koszcie.

Niech $\gamma$ będzie kosztem ścieżki przekształcającej $s$ w $t$. Rozważmy zbiór $A$ ścieżek edycyjnych, które przekształcają $s$ w $t$ i których koszt jest ograniczony z góry przez $\gamma$. Zbiór $A$ jest niepusty i składa się z operacji edycyjnych o dodatnim koszcie mniejszym niż $\gamma$. Zbiór operacji bazowych, których koszt jest ograniczony z góry przez $\gamma$ jest skończony, co dowodzi, że zbiór $A$ jest również skończony. Ponieważ $A$ jest niepusty i skończony, to ścieżki edycyjne o mininalnym (dodatnim) koszcie istnieją i należą do $A$. Stąd, $ED(s,t) > 0$ dla $s\neq t$, tj. odległość edycyjna jest dodatnio określona.

Aby udowodnić symetrię odległości edycyjnej, rozważmy optymalną ścieżkę $E$, która przekształca $s$ w $t$, oraz odpowiadającą jej odwrotną ścieżkę $E_r$, która przekształca $t$ w $s$. Równość ich kosztów $\delta(E) = \delta(E_r)$ wynika z symetrii funkcji kosztu i symetrii zbioru operacji bazowych~$\mathbb{B}$. 

Aby pokazać subaddytywność, rozważmy optymalną ścieżkę $E_1$, która przekształca $s$ w $t$, optymalną ścieżkę $E_2$, która przekształca $t$ w $u$, oraz złożenie ścieżek $E_1E_2$, które przekształca $s$ w $u$. Z tego, że $\delta(E_1E_2) = \delta(E_1) + \delta(E_2) = ED(s,t)+ED(t,u)$ oraz $\delta(E_1E_2) \geq ED(s,u)$ wynika, że $ED(s,t)+ED(t,u) \geq ED(s,u)$.
\end{proof}

Odległość edycyjna jest metryką, nawet gdy funkcja kosztu $\delta$ nie jest subaddytywna. Co więcej, ponieważ ciąg nakładających się operacji, które przekształcają $s$ w $t$, mogą mieć mniejszy koszt niż $\delta(s \rightarrow t)$, $\delta(s \rightarrow t)$ może być większe niż $ED(s,t)$. Rozważmy, na przykład, następujący alfbet: $\{\verb|a, b, c|\}$, gdzie symetria i brak subaddytywności funkcji $\delta$ jest zdefiniowana następująco:
\begin{align*}
\delta(\PVerb{a} \rightarrow \PVerb{c}) = \delta(\PVerb{b} \rightarrow \PVerb{c}) = 1 \\
\delta(\PVerb{a} \rightarrow \varepsilon) = \delta(\PVerb{b} \rightarrow \varepsilon) = \delta(\PVerb{c} \rightarrow \varepsilon) = 2 \\
\delta(\PVerb{a} \rightarrow \PVerb{b}) = 3
\end{align*}

Można zobaczyć, że $3 = \delta(\verb|a| \rightarrow \verb|b|) > \delta(\verb|a| \rightarrow \verb|c|) + \delta(\verb|c| \rightarrow \verb|b|) = 2$. Stąd optymalna ścieżka edycyjna $(\verb|a| \rightarrow \verb|c|, \verb|c| \rightarrow \verb|b|)$ przekształca $\verb|a|$ w $\verb|b|$ z kosztem równym $2$.

\textbf{Ścisła odległość edycyjna.} Subaddytywność odległości edycyjnej pozwala używać metod właściwych przestrzeniom metrycznym. Niemniej jednak, problem minimalizacji zbioru nakładających się operacji edycyjnych, może być trudny. Aby zrównoważyć złożoność obliczeniową, zazwyczaj używana jest funkcja podobieństwa, zdefiniowana jako minimum kosztu \emph{ścisłej ścieżki edycyjnej}. Ta ostatnia nie zawiera nakładających się na siebie operacji edycyjnych i nie modyfikuje tego samego podnapisu więcej niż raz. Odpowiająca jej odległość edycyjna nazywana jest \emph{ścisłą odległością edycyjną}~\cite{Boytsov2011:indexingmethods}.

\begin{lemma}
Dowolna nieścisła odległość edycyjna ogranicza z dołu odpowiadającą jej ścisłą odległość edycyjną.
\end{lemma}

\begin{lemma}
\label{lem:001}
Ścisła odległość Levenshteina o jednostkowym koszcie operacji jest równa nieścisłej odległości Levenshteina o jednostkowym koszcie operacji.
\end{lemma}

Powyższe wynika natychmiast z obserwacji, że optymalna ścieżka edycyjna zawiera jednoznakowe usunięcia, wstawienia oraz zamiany, które nigdy nie modyfikują znaku więcej niż raz.

\begin{lemma}
Nieścisła odległość Damerau-Levenshteina oraz ścisła odległość Damerau-Levenshteina są różnymi funkcjami. Co więcej ścisła odległość Damerau-Levenshteina nie jest metryką, gdyż nie jest subaddytywna~\cite{Boytsov2011:indexingmethods}.
\end{lemma}

\begin{proof}
Ścisła odległość Damerau-Levenshteina traktuje transpozycję (tj. zamianę dwóch przylegających do siebie znaków) jako bazową operację edycyjną. Aby udowodnić lemat podamy przykłąd, w którym zakaz modyfikacji znaków już stransponowanych odróżnia odległość Damerau-Levenshteina od ścisłej odległości Damerau-Levenshteina~\cite{Boytsov2011:indexingmethods}.
\end{proof}

Rozważmy napisy \verb|ab|, \verb|ba| oraz \verb|acb|. Z jednej strony, najkrótsza nieścisła ścieżka edycyjna, która przekształca \verb|ba| w \verb|acb|, tj. $(\verb|ba| \rightarrow \verb|ab|, \varepsilon \rightarrow \verb|c|)$ zawiera dwie operacje: najpierw, zamienia znaki \verb|a| i \verb|b|, a następnie wstawia \verb|c| pomiędzy nie. Zauważmy, że wstawienie przekształca już transformowany napis. Jednakowoż, jeśli kolejne przekształcenia tego samego podnapisu są wykluczone, to najkrótsza ścieżka edycyjna, która przekształca \verb|ba| w \verb|acb|, składa się z trzech operacji edycyjnych, np. $(\verb|ba| \rightarrow \varepsilon, \varepsilon \rightarrow \verb|b|)$. Stąd, nieścisła odległość edycyjna jest równa dwa, podczas gdy ścisła odległość wynosi trzy~\cite{Boytsov2011:indexingmethods}. 

Ścisła odległość Damerau-Levenshteina nie spełnia nierówności trójkąta, gdyż
$$
\verb|ba|  \xrightarrow[1]{transp.\ b\ i \ a} \verb|ab| + \verb|ab| \xrightarrow[1]{wst. c} \verb|acb|,
$$
natomiast
$$
\verb|ba|  \xrightarrow[1]{us.\ b} \verb|a| \xrightarrow[1]{wst.\ c} \verb|ac| \xrightarrow[1]{wst.\ b} \verb|acb|
$$
zatem
$$
2 = ED(\verb|ba|, \verb|ab|) + ED(\verb|ab|, \verb|acb|) \leq ED(\verb|ba|, \verb|acb|) = 3.
$$

Ponieważ ścisła i nieścisła odległość Damerau-Levenshteina są różnymi funkcjami, tę pierwszą nazywa się często \emph{odległością optymalnego dopasowania napisów}. Od tego momentu w niniejszej pracy ścisłą odległość Damerau-Levenshteina nazywamy odległością optymalnego dopasowania napisów, natomiast nieścisłą odległość Damerau-Levenshteina nazywamy odległością Damerau-Levenshteina~\cite{Loo2014:stringdist}.

\textbf{Optymalne dopasowanie.} % Problem optymalnego obliczenia odległości Damerau-Levenshteina został rozwiązany przez Lowrance'a i Wagnera~\cite{Wagner1975:extensionstring}. W dalszej części podajemy formalne definicje oraz przegląd klasycznych dynamicznych algorytmów wyliczających odległość Levenshteina.

Niech napisy $s$ i $t$ zostaną podzielone na tę samą liczbę, być może pustych, podnapisów: $s = s_1 s_2 \ldots s_l$ i $t = t_1 t_2 \ldots t_l$, takich, że $s_i \rightarrow t_i \in \mathbb{B}$. Co więcej, zakładamy, że $s_i$ i $t_j$ nie mogą być puste dla $i = j$. Mówimy, że ten podział definiuje \emph{dopasowanie} $A = (s_1 s_2\ldots s_l, t_1 t_2 \ldots t_l)$ pomiędzy napisami $s$ i $t$, w którym podnapis $s_i$ jest dopasowane do podnapisu $t_i$~\cite{Boytsov2011:indexingmethods}.

Dopasowanie reprezentuje ścisła ścieżkę edycyjną $E = s_1 \rightarrow t_1, s_2 \rightarrow t_2, \ldots, s_l \rightarrow t_l$. Definiujemy \emph{koszt dopasowania} $A$ jako koszt odpowiadającej mu ścieżki edycyjnej i oznaczamy przez $\delta(A)$:

\begin{equation}
\label{eq:003}
\delta(A) = \sum\limits_{i = 1}^{l} \delta(s_i \rightarrow t_i)
\end{equation}

\emph{Optymalne dopasowanie} to dopasowanie o najmniejszym koszcie.

PIEKNY RYSUNEK

\begin{example}
Przykład optymalnego dopasowania pomiędzy słowami \verb|foczka| i \verb|kozak| prezentuje rys. TUTAJNUMER. Odpowiadająca mu ścieżka edycyjna składa się z zamiany $\verb|f| \rightarrow \verb|k|$, usunięcia $\verb|c| \rightarrow \varepsilon$ oraz transpozycji $\verb|ka| \rightarrow \verb|ak|$.
\end{example}

Warto zauważyć, że istnieje różnowartościowe (1-1) mapowanie pomiędzy zbiorem ścisłych ścieżek edycyjnych i zbiorem dopasowań: każda ścisła ścieżka edycyjna o minimalnym koszcie reprezentuje dopasowanie o najmniejszym koszcie i odwrotnie. Stąd można zastąpić problem znalezienia optymalnej ścisłej odległości edycyjnej poprzez problem znalezienia optymalnego dopasowania.

\textbf{Obliczanie odległości edycyjnej.} Główną zasadą dynamicznego algorytmu, liczącego koszt optymalnego dopasowania, jest wyrażenie kosztu dopasowania pomiędzy napisami $s$ i $t$, używając kosztu dopasowaniu ich prefiksów. Rozważmy prefiks $s_{1:i}$ o długości $i$ i prefiks $t_{1:j}$ o długości $j$ napisów $s$ i $t$ odpowiednio. Załóżmy, że $A = (s_1 s_2\ldots s_l, t_1 t_2 \ldots t_l)$ jest optymalnym dopasowaniem pomiędzy $s_{1:i}$ i $t_{1:j}$, którego koszt oznaczamy przez $C_{i,j}$~\cite{Boytsov2011:indexingmethods}.

Używając równania \ref{eq:003} oraz definicji optymalnego dopasowania, łatwo pokazać, że $C_{i,j}$ może zostać policzone przy użyciu następującej ogólnej rekurencji~\cite{Ukkonen1985:algorithmsforapprox}:


\begin{align}
\label{eq:004}
C_{0,0} = 0 \\
C_{i,j} = min\{\delta(s_{i^\prime:i}\rightarrow t_{j^\prime:j}) + C_{i^\prime-1, j^\prime-1} | s_{i^\prime:i}\rightarrow t_{j^\prime:j} \in \mathbb{B}\} 
\end{align}

%Rekurencja \ref{eq:004} jest przykładem dynamicznego programowania. Zbiór liczb $\{C_{i,j}\}$ o liczności $(|s|+1)\cdot (|t|+1)$ jest zazwyczaj nazywany \emph{macierzą dynamicznego programowania} (lub w skrócie macierzą DP). Co więcej, 
Można zauważyć, że:
\begin{itemize}
\item koszt dopasowania napisów $s$ i $t$ jest równy $C_{|s|, |t|}$;
\item wszystkie optymalne dopasowania mogą zostać wyznaczone poprzez odwracanie rekurencji \ref{eq:004} (przechodzenie od tyłu).
\end{itemize}




Rozważmy teraz odległość Hamminga, gdzie $s_{i^\prime:i}\rightarrow t_{j^\prime:j}$ to zamiany znaków o koszcie równym jeden. Stąd,
\begin{equation}
\label{eq:005}
\delta(s_{i^\prime:i}\rightarrow t_{j^\prime:j}) = [s_{i^\prime:i}\neq t_{j^\prime:j}]
\end{equation}

gdzie $[X]$ jest równe jeden, gdy warunek $X$ jest spełniony, zero w przeciwnym przypadku. Co więcej, w tym przypadku możliwa jest tylko jedna kombinacja $i^\prime$ oraz $j^\prime$, mianowicie $i^\prime = i$ oraz $j^\prime = j$. Co więcej, odległość ta jest zdefiniowana jedynie dla $|s| = |t|$, zatem $C_{i,j}$ może być policzone jedynie dla $i = j$. Wówczas definicja odległości Hamminga nie jest rekursywna (rekursyjna?) i można ją zapisać następująco:


\begin{definition}
\emph{Odległością Hamminga} nazywamy~\cite{Hamming1950:errordetecting}:
$$
d_{hamming}(s, t) = \left\{
\begin{array}{l l}     
    \sum\limits_{i=1}^{|s|}\delta(s_i \rightarrow t_i) = \sum\limits_{i=1}^{|s|}[s_i\neq t_i], & \text{gdy } |s| = |t|,\\
    \infty, & \text{w przeciwnym przypadku},
\end{array}\right.
$$
%gdzie 
%$$
%\delta(s_i, t_i) = \left\{
%\begin{array}{l l}     
%    1, & \text{gdy } s_i = t_i,\\
%    0, & \text{w przeciwnym przypadku}.
%\end{array}\right.
%$$
\end{definition}

%Odległość Hamminga dopuszcza jedynie zamianę znaku, stąd jest zdefiniowana tylko dla napisów o~równej długości. 
%Łatwo zauważyć, że~odległość Hamminga spełnia definicję metryki. 
Odległość Hamminga zlicza liczbę indeksów, na~których dwa napisy mają różny znak. Odległość ta~przyjmuje wartości ze~zbioru $\{0,\ldots,|s|\}$, gdy $|s|=|t|$, natomiast jest równa nieskończoności, gdy napisy mają różne długości.

[PIĘKNY RYSUNEK??]

\begin{example}
Odległość Hamminga między słowami \verb|koza| i \verb|foka| wynosi $d_{hamming}(\verb|koza|, \verb|foka|) = 2$, natomiast między słowami \verb|koza| i \verb|foczka| wynosi ona $d_{hamming}(\verb|koza|, \verb|foczka|) = \infty$.
\end{example}




Rozważmy teraz odległość najdłuższego wspólnego podnapisu, gdzie $s_{i^\prime:i}\rightarrow t_{j^\prime:j}$ to wstawienia i usunięcia znaków o koszcie równym jeden. Wówczas istnieją dwie kombinacje $i^\prime$ oraz $j^\prime$ z równania \ref{eq:005}, odpowiadające usunięciu i wstawieniu, odpowiednio:
\begin{itemize}
\item $i^\prime = i - 1$ oraz $j^\prime = j$,
\item $i^\prime = i$ oraz $j^\prime = j - 1$.
\end{itemize}

Uwzględniając powyższe uproszczenia, możemy następująco przepisać ogólną postać rekurencji \ref{eq:004} dla odległości najdłuższego wspólnego podnapisu [NIE WIEM CZY TO JEST DOBRZE!!!!]:

\begin{equation}
C_{i,j} = min\left\{
\begin{array}{l l}     
    0, & \text{gdy } i = j = 0 \\
    C_{i-1, j} + 1, & \text{gdy } i > 0 \\
    C_{i, j-1} + 1, & \text{gdy } j > 0
\end{array}\right.
\end{equation}


%\begin{definition}
%\emph{Odległością najdłuższego wspólnego podnapisu}~\cite{Needleman2008:generalmethod} na $\Sigma^*$ nazywamy:
%$$
%d_{lcs}(s, t) = \left\{
%\begin{array}{l l}     
%    0, & \text{gdy } s = t = \varepsilon,\\
%    d_{lcs}(s_{1:|s|-1}, t_{1:|t|-1}), & \text{gdy } s_{|s|} = t_{|t|}, \\
%    1+min\{d_{lcs}(s_{1:|s|-1}, t), d_{lcs}(s, t_{1:|t|-1})\}, & \text{w przeciwnym przypadku},
%\end{array}\right.
%$$
%\end{definition}

Odległość najdłuższego wspólnego przyjmuje wartości ze zbioru $\{0, |s|+|t|\}$, przy czym maksimum jest osiągane, gdy $s$ i $t$ nie mają ani jednego wspólnego znaku. Odległość tę oznaczamy przez $d_{lcs}$.
%Odległość ta zlicza liczbę usunięć i~wstawień, potrzebnych do~przetworzenia jednego napisu w~drugi. 

\begin{example}
Odległość najdłuższego wspólnego podnapisu między napisami \verb|koza| i \verb|foka| wynosi: $d_{lsc}(\verb|koza|, \verb|foka|) = 4$, bo~$\verb|koza|\xrightarrow[1]{us.\ k} \verb|oza| \xrightarrow[1]{us.\ z} \verb|oa|  \xrightarrow[1]{wst.\ f} \verb|foa| \xrightarrow[1]{wst.\ k} \verb|foka|$.
\end{example}

Powyższy przykład pokazuje, że~w~ogólności nie ma~unikalnej najkrótszej drogi transformacji jednego napisu w~drugi, gdyż można zamienić kolejność usuwania (lub wstawiania) znaków i~również uzyskać odlegość równą~$4$.

Jak sugeruje nazwa, odległość najdłuższego wspólnego podnapisu, ma~też inną interpretację. Poprzez wyrażenie \emph{najdłuższy wspólny podnapis} rozumiemy najdłuższy ciąg utworzony przez sparowanie znaków z~$s$~i~$t$~nie zmieniając ich porządku. Wówczas odległość ta~jest rozumiana jako liczba niesparowanych znaków z~obu napisów. W~powyższym przykładzie może to~być zwizualizowane następująco:
	
[PIĘKNY RYSUNEK]

Jak widać na~rysunku, znaki \verb|k|, \verb|z|, \verb|f| i \verb|k| pozostają bez pary, dając odległość równą~$4$.


Przejdźmy do odległości Levenshteina. Odległość ta dopuszcza, oprócz usunięć i wstawień, także zamiany znaków. Istnieją zatem  trzy kombinacje $i^\prime$ oraz $j^\prime$ z równania \ref{eq:005}:
\begin{itemize}
\item $i^\prime = i - 1$ oraz $j^\prime = j$,
\item $i^\prime = i$ oraz $j^\prime = j - 1$,
\item $i^\prime = i - 1$ oraz $j^\prime = j - 1$.
\end{itemize}

Stąd ogólna postać rekurencji \ref{eq:004} dla odległości Levenshteina może zostać przepisana następująco:

\begin{equation}
\label{eq:006}
C_{i,j} = min\left\{
\begin{array}{l l}     
    0, & \text{gdy } i = j = 0\\
    C_{i-1, j} + 1, & \text{gdy } i > 0 \\
    C_{i, j-1} + 1, & \text{gdy } j > 0 \\
    C_{i-1, j-1} + [s_{i}\neq t_{j}], & \text{gdy } i, j > 0
\end{array}\right.
\end{equation}

Odległość Levenshteina oznaczamy przez $d_{lv}$.

%\begin{definition}
%Uogólnioną \emph{odległością Levenshteina} \cite{Levenshtein1965:binarycodes}~na~$\Sigma^*$~nazywamy:
%$$
%d_{lv}(s, t) = \left\{
%\begin{array}{l l}     
%    0, & \text{gdy } s = t = \varepsilon,\\
%    min\{ & \\
%\qquad    d_{lv}(s, t_{1:|t|-1}) + w_1, & \\
%\qquad    d_{lv}(s_{1:|s|-1}, t) + w_2, & \\
%\qquad    d_{lv}(s_{1:|s|-1}, t_{1:|t|-1}) + [1-\delta(s_{|s|}, t_{|t|})]w_3 & \\
%\qquad    \}, & \text{w przeciwnym przypadku},
%\end{array}\right.
%$$
%gdzie $w_1, w_2$~i~$w_3$~to~niezerowe liczby rzeczywiste, oznaczające kary za~usunięcie, wstawienie oraz zamianę znaku.
%\end{definition}

%Odległość ta~zlicza ważoną sumę usunięć, wstawień oraz zamian znaków, potrzebnych do~przetworzenia jednego napisu w~drugi. Gdy za~wagi przyjmie się $1$~mamy do~czynienia ze~zwykłą odległością Levenshteina, np.~

\begin{example}
Odległość Levenshteina między napisami \verb|koza| i \verb|foka| wynosi: $d_{lv}(\verb|koza|, \verb|foka|) = 2$, bo $\verb|koza| \xrightarrow[1]{zm.\ k\ na\ f} \verb|foza| \xrightarrow[1]{zm.\ z\ na\ k} \verb|foka|$.
\end{example}

Powyższy przykład ilustruje, że dodatkowa elastyczność w~porównaniu do~odległości najdłuższego wspólnego podnapisu, daje mniejszą wartość odległości między napisami, jako że~potrzebujemy jedynie dwóch zamian znaków~\cite{Loo2014:stringdist}.

Przypomnijmy, że mówimy o uogólnionej odległości, gdy zmienimy koszty poszczególnych operacji na różne od jeden.
Gdy za~koszt przyjmiemy np. $(0.1, 1, 0.3)$ dla usunięć, wstawień i zamian znaków odpowiednio, to uogólniona odległość Levenshteina między napisami \verb|koza| i \verb|foka| wynosi: $d_{lv}(\verb|koza|, \verb|foka|) = 0.6$, bo~$\verb|koza|  \xrightarrow[0.3]{zm.\ k\ na\ f} \verb|foza|  \xrightarrow[0.3]{zm.\ z\ na\ k} \verb|foka|$.

Uogólniona odległość Levenshteina spełnia definicję metryki, gdy koszt usunięcia jest równy kosztowi wstawienia znaku. W przeciwnym przypadku nie spełnia ona założenia o~symetrii. Jednakowoż, symetria zostaje zachowana przy jednoczesnej zamianie $s$~i~$t$~oraz kosztów usunięcia i wstawienia znaku,~jako że~liczba usunięć znaków przy przetwarzaniu napisu $s$~w~napis $t$~jest równa liczbie wstawień znaków przy przetwarzaniu napisu $t$~w~napis $s$~\cite{Loo2014:stringdist}.~Dobrze obrazuje to~następujący przykład.

\begin{example}
Przyjmijmy za~koszt usunięcia, wstawienia i zamiany znaku $(0.1, 1, 0.3)$ odpowiednio. Wówczas uogólniona odległość Levenshteina dla napisów \verb|koza| i~\verb|foczka| wynosi:
\begin{equation}
\label{eq:001}
d_{lv}(\verb|koza|, \verb|foczka|) = 0.5,
\end{equation}
bo
$$
\verb|koza|  \xrightarrow[0.3]{zm.\ k\ na\ f} \verb|foza|  \xrightarrow[0.1]{wst. c} \verb|focza| \xrightarrow[0.1]{wst. k} \verb|foczka|,
$$
natomiast
\begin{equation}\label{eq:002}
d_{lv}(\verb|foczka|, \verb|koza|) = 2.3,
\end{equation}
bo
$$
\verb|foczka|  \xrightarrow[0.3]{zm.\ f\ na\ k} \verb|koczka|  \xrightarrow[1]{us. c} \verb|kozka| \xrightarrow[1]{us. k} \verb|koza|.
$$
Gdy za~koszty przyjmiemy $(1, 0.1, 0.3)$, to~uogólniona odległość Levenshteina wynosi:
$$
d_{lv}(\verb|koza|, \verb|foczka|) = 2.3,
$$
bo
$$
\verb|koza|  \xrightarrow[0.3]{zm.\ k\ na\ f} \verb|foza|  \xrightarrow[1]{wst. c} \verb|focza| \xrightarrow[1]{wst.\ k} \verb|foczka|,
$$
czyli analogicznie, jak w~przypadku \ref{eq:002}. Natomiast
$$
d_{lv}(\verb|foczka|, \verb|koza|) = 0.5,
$$
bo
$$
\verb|foczka|  \xrightarrow[0.3]{zm.\ f\ na\ k} \verb|koczka|  \xrightarrow[0.1]{us. c|} kozka \xrightarrow[0.1]{us.\ k} \verb|koza|,
$$
czyli analogicznie, jak w~przypadku \ref{eq:001}.
\end{example}

Zgodnie z lematem \ref{lem:001} nieścisła odległość Levenshteina jest równa ścisłej odległości Levenshteina. Z drugiej strony, ścisła odległość edycyjna jest równa kosztowi optymalnego dopasowania. Stąd rekurencja \ref{eq:004} liczy nieścisłą odległość Levenshteina. Częstym błędem jest, że następujące bezpośrednie uogólnienie rekurencji \ref{eq:006} liczy (nieścisłą) odległość Damerau-Levenshteina~\cite{Boytsov2011:indexingmethods}:

\begin{equation}
\label{eq:007}
C_{i,j} = min\left\{
\begin{array}{l l}     
    0, & \text{gdy } i = j = 0\\
    C_{i-1, j} + 1, & \text{gdy } i > 0 \\
    C_{i, j-1} + 1, & \text{gdy } j > 0 \\
    C_{i-1, j-1} + [s_{i}\neq t_{j}], & \text{gdy } i, j > 0 \\
    C_{i-2, j-2} + 1, & \text{gdy } s_i = t_{j-1}, s_{i-i} = t_j \text{ oraz } i,j > 1
\end{array}\right.
\end{equation}

Jednak, rekurencja \ref{eq:007} liczy odległość optymalnego dopasowania napisów, czyli ścisłą odległość Damerau-Levenshteina, która nie zawsze jest równa odległości Damerau-Levenshteina. Dla przykładu, odległość między napisami \verb|ba| i \verb|acb| wyliczona przy pomocy rekurencji \ref{eq:007} jest równa trzy, natomiast odległość Damerau-Levenshteina między tymi napisami wynosi dwa.


%
%\begin{definition}
%\emph{Odległością optymalnego dopasowania napisów} na~$\Sigma^*$~nazywamy:
%$$
%d_{osa}(s, t) = \left\{
%\begin{array}{l l}     
%    0, & \text{gdy } s = t = \varepsilon,\\
%    min\{ & \\
%\qquad    d_{osa}(s, t_{1:|t|-1}) + w_1, & \\
%\qquad    d_{osa}(s_{1:|s|-1}, t) + w_2, & \\
%\qquad    d_{osa}(s_{1:|s|-1}, t_{1:|t|-1}) + [1-\delta(s_{|s|}, t_{|t|})]w_3 & \\
%\qquad    d_{osa}(s_{1:|s|-2}, t_{1:|t|-2}) + w_4\text{, gdy } s_{|s|} = t_{|t|-1}, s_{|s|-1} = t_{|t|} & \\
%\qquad    \}, & \text{w przeciwnym przypadku},
%\end{array}\right.
%$$
%gdzie $w_1, w_2$, $w_3$~i$w_4$~to~niezerowe liczby rzeczywiste, oznaczające kary za~odpowiednio usunięcie, wstawienie, zamianę oraz transpozycję znaków.
%\end{definition}

%Odległość optymalnego dopasowania napisów jest bezpośrednim rozszerzeniem odległości Levenshteina, która zlicza również liczbę transpozycji przylegających znaków, potrzebnych do~przetworzenia jednego napisu w~drugi. W~przeciwieństwie do~wcześniej zaprezentowanych odległości, nie spełnia ona nierówności trójkąta, tj. podpunktu \ref{def:001d} z~definicji \ref{def:001} \cite{Loo2014:stringdist}: 
%$$2 = d_{osa}(\verb|ba|, \verb|ab|) + d_{osa}(\verb|ab|, \verb|acb|) \leq d_{osa}(\verb|ba|, \verb|acb|) = 3,$$ 
%gdyż
%$$
%\verb|ba|  \xrightarrow[1]{transp.\ b\ i \ a} \verb|ab| + \verb|ab| \xrightarrow[1]{wst. c} \verb|acb|,
%$$
%natomiast
%$$
%\verb|ba|  \xrightarrow[1]{us.\ b} \verb|a| \xrightarrow[1]{wst.\ c} \verb|ac| \xrightarrow[1]{wst.\ b} \verb|acb|.
%$$
%W ostatnim przykładzie, zmniejszenie odległości poprzez zamianę liter $|a|$ i $|b|$, a następnie wstwienie litery $|c|$ spowodowałoby dwukrotne przekształcenie tego samego podnapisu. Z tego powodu odległość optymalnego dopasowania napisów bywa również nazywana \emph{ścisłą odległością Damerau-Levenshteina} i jest często mylona z właściwą \emph{odległością Damerau-Levenshteina}. Ta ostatnia pozwala na przekształcanie tego samego podnapisu wielokrotnie i jest metryką w rozumieniu definicjij \ref{def:001}, ale nie spełnia założenia o nie przekształcaniu wielokrotnie tego samego podnapisu \cite{Loo2014:stringdist}.


[MIARA DAMERAU-LEVENSHTEINA]

W~przypadku odległości Levenshteina i~odległości optymalnego dopasowania napisów, maksymalna odległość między napisami $s$~i~$t$~wynosi $max\{|s|, |t|\}$. Jednakowoż, gdy liczba dopuszczalnych operacji edycyjnych rośnie, to liczba dopuszczalnych ścieżek między napisami wzrasta, co pozwala ewentualnie zmiejszyć odległość między napisami. Dlatego relację między zaprezentowanymi powyżej odległościami można podsumować następująco~\cite{Loo2014:stringdist}:
$$
\left. \begin{array}{r}
\infty \geq |s| \geq d_{hamming}(s,t) \\
|s| + |t| \geq d_{lcs}(s,t) \\
max\{|s|, |t|\} \\
\end{array} \right \}
\geq d_{lv}(s,t) \geq d_{osa}(s,t) \geq 0.
$$

%-----------Koniec części zasadniczej-----------

%\begin{thebibliography}{11}
%\nocite{Hornik2012:sphkmeans}
%\nocite{Wild2002:sphkmeans}
%\nocite{Loo2014:stringdist}
\bibliographystyle{plain}
\bibliography{bibliography}
%\bibitem[1]{M} Arnold W. Miller, \emph{Special Subsets of the Real Line}, The University of Texas, Austin, U.S.A. 1984.
%\bibitem[2]{H2} J.C. Oxtoby, \emph{Measure and Category}, Springer-Verlag, New York Heidelberg Berlin, 1971.
%\bibitem[3]{WW} Winfried Just and Martin Weese, \emph{Discovering Modern Set Theory. I: The Basics.} Graduate Studies in Mathematics vol. 8, American Mathematical Society, Providence, RI, 1996.
%\bibitem[4]{WW2} Winfried Just and Martin Weese, \emph{Discovering Modern Set Theory. II: Set-Theoretic Tools for Every Mathematician.}, Graduate Studies in Mathematics vol. 18, American Mathematical Society, Providence, RI, 1997.
%\end{thebibliography}
%\clearpage
%\pagestyle{empty}
%\noindent Warszawa, dnia ...............
%\vspace{5cm}
%\begin{center}
%\LARGE{Oświadczenie}
%\end{center}
%Oświadczam, że pracę licencjacką pod tytułem: ,,Automatyczna kategoryzacja tematyczna tekstów przy użyciu metryk w przestrzeni ciągów znaków'', której promotorem jest dr hab. Marek Gągolewski, wykonałem/am samodzielnie, co poświadczam własnoręcznym podpisem.
%\vspace{2cm}
%\begin{flushright}
%...........................................
%\end{flushright}

%\makestatement
\end{document}
