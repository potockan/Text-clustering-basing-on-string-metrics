\documentclass[12pt, twoside, openany]{report}
\usepackage[dvips]{rotating}
\usepackage[utf8]{inputenc}
\usepackage{t1enc}
\usepackage{a4wide}
\usepackage{amsfonts}
\usepackage{amsmath}
\usepackage{enumerate}
\usepackage{verbatim}
\usepackage[MeX]{polski}
\usepackage[T1]{fontenc}
\usepackage{geometry}
\geometry{left=25mm,right=25mm,%
bindingoffset=10mm, top=25mm, bottom=25mm}
\usepackage{amssymb, latexsym}
\usepackage{amsthm}
\usepackage{palatino}
\usepackage{array}
\usepackage{pstricks}
\usepackage{textcomp}
\usepackage{hyperref}
%paginy
\usepackage{fancyhdr}
\pagestyle{fancy}
% zmiana liter w~.ywej paginie na ma.e
\renewcommand{\chaptermark}[1]{\markboth{#1}{}}
\renewcommand{\sectionmark}[1]{\markright{\thesection\ #1}}
\fancyhf{} % usu. bie..ce ustawienia pagin
\fancyhead[LE,RO]{\small\bfseries\thepage}
\fancyhead[LO]{\small\bfseries\rightmark}
\fancyhead[RE]{\small\bfseries\leftmark}
\renewcommand{\headrulewidth}{0.5pt}
\renewcommand{\footrulewidth}{0pt}
\addtolength{\headheight}{0.5pt} % pionowy odst.p na kresk.
\fancypagestyle{plain}{%
\fancyhead{} % usu. p. górne na stronach pozbawionych
% numeracji (plain)
\renewcommand{\headrulewidth}{0pt} % pozioma kreska
}
%paginy koniec
\theoremstyle{plain}
\newtheorem{tw}{Twierdzenie}[section]
\newtheorem{uw}{Uwaga}[section]
\newtheorem{defi}{Definicja}[section]
\newtheorem{alg}{Algorytm}[chapter]
\newtheorem{przyp}{Przypadek}[section]
\newtheorem{prz}{Przykład}[section]
\newtheorem{lem}[tw]{Lemat}
\newtheorem{stw}[tw]{Stwierdzenie}
\newtheorem{wn}[tw]{Wniosek}
\newtheorem{cw}{Ćwiczenie}[section]
\linespread{1.5}
\author{Natalia Potocka}
\title{Słowa - statystyki}
\begin{document}
\maketitle
\hyphenation{Syl-ves-tra}
\hyphenation{Syl-ves-ter-a}
\newpage






W polskiej Wikipedii znaleziono 2867036 unikalnych słów. 50.11\% z nich występuje dokładnie w jednym tekście, natomiast 44.93\% występuje tylko raz. Dobrze obrazuje to poniższy histogram.

\begin{knitrout}
\definecolor{shadecolor}{rgb}{0.969, 0.969, 0.969}\color{fgcolor}
\includegraphics[width=\maxwidth]{figure/hist-1} 

\end{knitrout}

Słowa występujące tylko w jednym tekście to zazwyczaj liczby oraz słowa w obcych językach, przykładowo:

\begin{knitrout}
\definecolor{shadecolor}{rgb}{0.969, 0.969, 0.969}\color{fgcolor}\begin{kframe}
\begin{verbatim}
 [1] "pikosatelitami" "mahiane"        "zaindeksowana"  "uchybiają"     
 [5] "omiécourt"      "nittle"         "clériga"        "mozon"         
 [9] "jūhuì"          "wardz"         
\end{verbatim}
\end{kframe}
\end{knitrout}

choć czasem są to słowa będące odmianą słów bardziej częściej występujących, np. słowo \emph{uchybiają} jest odmianą czasownika \emph{uchybiać}. Należy wziąć pod uwagę takie słowa, gdyż mogą one polepszyć jakość dopasowania tekstów pod względem tematycznym.

% Po usunięciu słów występujących w jednym, dwóch lub trzech tekstach otrzymujemy następujące statystyki:
%
% <<stat_3, dependson='ws'>>=
% cnt <- 3
% stat_cnt <- which(word_stat$word_cnt>cnt)
% print(summary(word_stat$word_cnt[stat_cnt]))
% n_stat_3 <- length(stat_cnt)/n_words*100
% @
%
%
% Słowa, które występują przynajmniej w czterech stanowią zaledwie n_stat_3\% wszystkich słów.

Słowa występujące w największej liczbie tekstów to:\\

% latex table generated in R 3.1.2 by xtable 1.7-4 package
% Sat Jan  3 11:13:11 2015
\begin{tabular}{rlrr}
  \hline
 & Słowo & Liczba artykułów & Liczba wystąpień słowa \\ 
  \hline
1 & w & 1001691 & 10266388 \\ 
  2 & i & 725134 & 4260343 \\ 
  3 & na & 687524 & 3185292 \\ 
  4 & z & 647672 & 3223226 \\ 
  5 & do & 559857 & 2280198 \\ 
  6 & się & 535836 & 2070729 \\ 
  7 & a & 419801 & 1066807 \\ 
  8 & roku & 419064 & 1285444 \\ 
  9 & od & 377392 & 929421 \\ 
  10 & jest & 342071 & 977477 \\ 
   \hline
\end{tabular}



%%%%%%%%%%%%%%%%% TO DO %%%%%%%%%%%%%%%%%%%%%%
%%%% wgrac (slowo, licznosc slowa ogolem) %%%%
W większości przypadków są to słowa nie istotne w kontekście analizy tematycznej tekstu. Podobnych słów jest więcej. Ich pełna lista zostaje przedstawiona poniżej:
\begin{knitrout}
\definecolor{shadecolor}{rgb}{0.969, 0.969, 0.969}\color{fgcolor}\begin{kframe}
\begin{verbatim}
  [1] "ach"         "aj"          "albo"        "bardzo"      "bez"        
  [6] "bo"          "być"         "ci"          "cię"         "ciebie"     
 [11] "co"          "czy"         "daleko"      "dla"         "dlaczego"   
 [16] "dlatego"     "do"          "dobrze"      "dokąd"       "dość"       
 [21] "dużo"        "dwa"         "dwaj"        "dwie"        "dwoje"      
 [26] "dziś"        "dzisiaj"     "gdyby"       "gdzie"       "go"         
 [31] "ich"         "ile"         "im"          "inny"        "ja"         
 [36] "ją"          "jak"         "jakby"       "jaki"        "je"         
 [41] "jeden"       "jedna"       "jedno"       "jego"        "jej"        
 [46] "jemu"        "jeśli"       "jest"        "jestem"      "jeżeli"     
 [51] "już"         "każdy"       "kiedy"       "kierunku"    "kto"        
 [56] "ku"          "lub"         "ma"          "mają"        "mam"        
 [61] "mi"          "mną"         "mnie"        "moi"         "mój"        
 [66] "moja"        "moje"        "może"        "mu"          "my"         
 [71] "na"          "nam"         "nami"        "nas"         "nasi"       
 [76] "nasz"        "nasza"       "nasze"       "natychmiast" "nią"        
 [81] "nic"         "nich"        "nie"         "niego"       "niej"       
 [86] "niemu"       "nigdy"       "nim"         "nimi"        "niż"        
 [91] "obok"        "od"          "około"       "on"          "ona"        
 [96] "one"         "oni"         "ono"         "owszem"      "po"         
[101] "pod"         "ponieważ"    "przed"       "przedtem"    "są"         
[106] "sam"         "sama"        "się"         "skąd"        "tak"        
[111] "taki"        "tam"         "ten"         "to"          "tobą"       
[116] "tobie"       "tu"          "tutaj"       "twoi"        "twój"       
[121] "twoja"       "twoje"       "ty"          "wam"         "wami"       
[126] "was"         "wasi"        "wasz"        "wasza"       "wasze"      
[131] "we"          "więc"        "wszystko"    "wtedy"       "wy"         
[136] "żaden"       "zawsze"      "że"         
\end{verbatim}
\end{kframe}
\end{knitrout}

\begin{knitrout}
\definecolor{shadecolor}{rgb}{0.969, 0.969, 0.969}\color{fgcolor}\begin{kframe}


{\ttfamily\noindent\bfseries\color{errorcolor}{Error in as.vector(x): object 'stat\_cnt' not found}}

{\ttfamily\noindent\bfseries\color{errorcolor}{Error in as.data.frame(y): object 'words' not found}}

{\ttfamily\noindent\bfseries\color{errorcolor}{Error in nrow(words\_no\_stp\$words): object 'words\_no\_stp' not found}}\end{kframe}
\end{knitrout}



